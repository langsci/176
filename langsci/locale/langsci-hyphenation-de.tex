\hyphenation{%
Aar-au
ab-ge-kürzt
ab-ge-lei-tet
Ab-lei-ten
ab-zu-lei-ten
a-dä-qua-ter
A-djek-tiv-fle-xi-on
Ad-res-sa-ten
Adri-ano
affri-ca-te
affri-ca-tes
Ak-ku-sa-tiv
A-le-man-ni-schen
all-ge-mein-ste
Alt-hoch-deutsch
Alt-hoch-deut-sche
Alt-hoch-deut-schen
A-na-lo-gi-en
A-na-ly-se
a-na-ly-siert
an-ge-nom-men
an-nä-hernd
An-satz
an-setzt
An-tje
an-zu-füh-ren
Ar-bei-ten
Ar-ti-ku-la-tion
Auf-tre-ten
auf-wach-sen
auf-weist
Aus-blei-chung
aus-ei-nan-der-ge-hal-ten
aus-fal-len
aus-führ-lich
aus-führ-li-che
Aus-gleich
Aus-gleichs-me-cha-nis-men
Aus-gleichs-ten-denz
Aus-gleichs-ten-den-zen
aus-lau-ten-de
aus-schließ-lich
Aus-schnitt
Bän-de
ba-sie-end
Be-deu-tungs-as-pekt
Be-deu-tungs-as-pekt-en
be-dingt
be-ding-te
Bei-spiel
Bei-spie-le
bei-spiels-wei-se
bei-spiels-wei-se
Be-lebt-heit
Be-lebt-heits-hie-rar-chie
be-nö-tigt
be-ob-ach-ten
be-ob-ach-tet
Be-ob-ach-tung
Be-reich
Be-rei-che
Be-reichs
be-reits
Be-rich-ti-gun-gen
bern-deut-sches
be-schäf-tigt
Be-schrän-kung
Be-schränk-ung-en
be-schreibt
Be-schrei-bung
be-schrie-ben
bes-ser
be-steht
be-stimmt
be-stimm-te
be-stimm-ten
be-stimm-ter
Be-tont-heit
be-trach-tet
be-tref-fend
Be-völ-ke-rungs-dich-te
Be-völ-ke-rungs-grö-ße
be-zie-hen
be-züg-lich
Blö-cke
Blö-cken
Bo-den-see-a-le-man-nisch
bräuch-te
bün-dig
Cha-rak-te-ri-sti-ka
Cha-rak-te-ris-ti-ka
Char-les
com-ple-ments
con-straint
da-für
Dam-mel
da-ran
da-von
da-zu
de-fi-nie-ren
de-fi-niert
De-mon-stra-tiv-pro-no-men
De-mon-stra-tiv-pro-no-mens
des-sen
des-we-gen
De-ter-mi-nie-rer
De-ter-mi-nie-rer-ka-te-go-rie-en
De-ter-mi-nie-rern
deut-sche
deut-schen
di-a-chron
di-a-chrone
di-a-chro-ner
Di-a-lekt-grup-pe
die-nen
dient
die-sel-be
die-sem
die-sen
die-ser
Die-wald
Diph-thon-gie-rung
Dis-kurs-kon-text
Dis-kurs-teil-neh-mer
Dis-kurs-teil-neh-mers
Dis-kurs-ver-lauf
Dis-kurs-ver-läu-fe
Di-stinkt-heit
durch-aus
Durch-schnitt-li-che
durch-schnitt-li-chen
Durch-schnitts-wer-te
durch-zu-füh-ren
Dy-na-mi-ken
dy-na-mi-sche
Ei-gen-schaf-ten
ei-nem
ei-ner
ein-fa-cher
ein-ge-gan-gen
ein-ge-stie-gen
ein-ge-teilt
ein-heit-li-ches
Ein-tei-lung
E-le-men-te
E-li-sa-beth-tal
el-säs-si-scher
ent-hält
ent-spre-chen
ent-we-der
ent-wi-ckeln
er-klä-ren
er-klärt
Er-klär-ung
er-mög-li-chen
er-rei-chen
er-scheint
ers-ten
Ers-tens
er-wei-tert
Eu-ro-pe
e-ven-tu-el-le
exem-pla-risch
Fak-to-ren
fal-len
feh-len-de
fest-hal-ten
fin-det
Fin-kel
Fle-xi-on
Fle-xi-ons-ka-te-go-ri-en
Fle-xi-ons-klas-se
Fle-xi-ons-klas-sen
Fle-xi-ons-mor-pho-lo-gie
Fle-xi-ons-pa-ra-dig-ma
Fle-xi-ons-suf-fix
Fle-xi-ons-sys-stem
Fol-ge
folg-lich
Folg-lich
folgt
Frank-reichs
fran-zö-sisch
fran-zö-si-schen
Frei-burg
Fri-bourg
früh-neu-hoch-deut-sche
fünf-tens
Funk-ti-ons-glei-chung
Gebiet
Ge-brauchs-un-ter-schied
ge-folgt
ge-fragt
ge-ge-ben
Ge-gen-teil
ge-gen-über
ge-gen-ü-ber-ge-stellt
ge-glie-dert
ge-hö-ren
ge-macht
Ge-mein-de
ge-mein-sa-men
ge-meint
ge-nannt
ge-nann-ten
ge-nau-er
ge-ne-ra-ti-ven
ge-ne-rell
Ge-ni-tiv
Ge-ni-tivs
Ge-nus
Ge-nus-syn-kre-tis-men
ge-o-gra-fisch
ge-o-gra-fi-sche
ge-prüft
ge-ra-de
ge-rings-te
ger-ma-ni-schen
Ge-samt-kom-ple-xi-tät
Ge-schlecht
ge-spei-chert
Ge-sprächs-part-ner
Ge-sprächs-part-ners
Ge-sprächs-teil-neh-mer
ge-spro-chen
ge-spro-che-nen
ge-wor-fen
ge-zählt
ge-zeigt
gleich
glei-cher
Gleich-zei-tig
gram-ma-ti-ka-li-sier-ten
Gram-ma-ti-ka-li-sie-rung
Grenz-wert
gro-ße
grund-sätz-lich
grund-sätz-li-che
Gruy-ter
Gur-mels
ha-ben
han-delt
Haupt-ar-gu-ment
Hebrä-isch
Hebrä-ische
Hebrä-ischen
He-ran-ge-hens-wei-sen
her-zu-lei-ten
he-te-ro-ge-nen
hier
hie-rar-chi-schen
hier-hin
hie-si-gen
hin-ge-wie-sen
his-to-risch
his-to-rische
Hoch-a-le-man-nisch
hoch-ale-man-ni-sche
Hoch-ale-man-ni-sche
Hoch-a-le-man-ni-schen
hoch-deut-sche
Höchst-a-le-man-nisch
höchst-a-le-man-ni-sche
höchst-ale-man-ni-schen
Höchst-ale-man-ni-schen
hö-he-re
hö-he-rem
hö-he-ren
hö-he-rer
Ho-mo-ge-ni-tät
Hö-rer
Hot-zen-kö-cher-le
Hu-zen-bach
Iko-ni-zi-tät
in-for-ma-ti-ons-the-o-re-ti-sches
in-for-mel-le
In-no-va-ti-ons-ra-te
In-no-va-ti-vi-tät
ins-be-son-de-re
in-te-res-sant
in-te-res-san-ter-wei-se
In-te-res-ses
in-te-res-sier-ten
In-ter-ro-ga-tiv-pro-no-men
In-ter-ro-ga-tiv-pro-no-mens
i-so-liert
i-so-lier-te
i-so-lier-ten
Jahr-hun-dert
je-der
je-ner
je-weils
Jid-disch
Jid-disch-en
Jung-gram-ma-ti-kern
Kai-ser-stuhl
Ka-pi-tel
Ka-pi-teln
Ka-pi-tels
Karl-Ham-pus
Ka-sus-mar-kie-rung
Ka-te-go-rie
Ka-te-go-ri-en
Ka-te-go-ri-sie-rung
kei-nem
Kir-chen-sla-vi-schen
klein-ste
ko-diert
kom-ple-xer
Kom-ple-xi-fi-zie-rung
Kom-ple-xi-tät
Kom-ple-xi-täts-aus-gleich
Kom-pri-mie-rung
kon-kre-ten
kon-kre-ter
kon-kur-rie-ren
kon-kur-riert
konn-te
Kon-sti-tu-en-ten-struk-tur
kons-ti-tu-tio-nel-ler
Kon-tak-te
Kon-takt-spra-chen
ko-o-pe-ra-ti-ven
kor-pus-ba-sier-te
Kürsch-ner
lang-sa-mer
Laut-qua-li-tät
leicht
Letz-te-re
Le-xems
Lexi-kon
lie-gen
lin-gu-is-ti-schen
Li-te-ra-tur-hebrä-isch-en
Mann-heim
mar-kiert
Mar-kie-rung
Mas-ku-lin
ma-xi-mal
Me-cha-nis-men
meh-re-ren
Mehr-spra-chig-keit
mesch-li-chen
Mess-me-tho-de
Mess-me-tho-den
Me-tho-de
Me-tho-den
Mi-ni-ma-lis-mus
mit-ei-nan-der
Mit-tel-hoch-deutsch
Mit-tel-hoch-deut-sche
Mit-tel-hoch-deut-schen
Mit-tel-sil-ben-zen-tra-li-sier-ung
Mo-bi-li-tät
Mo-dell
mo-del-liert
mo-der-nen
Mo-di-fi-ka-ti-on
Mo-du-le
mög-li-che
mög-li-chen
mög-li-cher
mög-lichst
Mo-ment
Mon-go-lisch
Mo-no-fle-xi-on
mor-pho-lo-gisch
mor-pho-lo-gi-sche
mor-pho-lo-gi-schen
mor-pho-syn-tak-ti-sche
mor-pho-syn-tak-ti-schen
mo-vier-ten
mul-ti-lin-gu-a-len
Mund-art
münd-lich
müs-sen
nach-ge-wie-sen
na-sa-lier-ten
ne-ben
Ne-ben-sil-be
Ne-ben-sil-ben-schwä-chung
Netz-werk
Netz-wer-ken
Neu-e-run-gen
Neu-trum
Nie-der-a-le-man-nisch
nie-der-a-le-man-ni-schen
nie-dri-ge-ren
Nie-mey-er
No-mi-nal-fle-xi-on
No-mi-nal-phra-se
No-mi-na-tiv
o-ben
Ober-fläch-en-ab-fol-ge
O-ber-rhein-a-le-man-nisch
o-ber-rhein-a-le-man-ni-sche
O-ber-rhein-a-le-man-ni-schen
öko-no-mi-scher
on-to-lo-gi-sche
O-pa-zi-tät
o-pe-ra-ti-o-na-li-siert
Ope-ra-ti-o-na-li-sie-rung
or-ga-ni-siert
or-na-men-ta-le
Ost-jid-disch-en
Pa-ra-dig-ma
Pa-ra-dig-men
Par-ti-zip
Pe-ri-phra-sen
pe-ri-phras-ti-scher
Per-so-nal-pro-no-men
Pe-tri-feld
Phä-no-me-ne
phi-lo-so-phi-schen
pho-no-lo-gisch
pho-no-lo-gi-sche
pho-no-lo-gi-schen
plä-diert
Po-si-tions-mög-lich-keit
Po-si-tions-mög-lich-keit-en
pos-se-si-ven
Pos-ses-siv-pro-no-men
Pos-ses-siv-pro-no-mens
prag-ma-ti-sch-en
prak-ti-schen
prä-sen-tiert
prä-zi-se
Pri-mär
Pri-mär-
Pri-mär-um-laut
prin-zi-pi-el-len
Pro-zes-se
Qua-li-tät
Qua-li-täts-un-ter-schie-de
re-a-li-sie-ren-de
re-a-li-sie-rung
Re-a-li-sie-rungs-re-geln
Rechts-att-ri-bu-te
Re-dun-dan-zen
re-gel-mä-ßi-gen
Re-geln
re-la-ti-ver
Rhein-ebe-ne
Rot-weil
Schlach-ter
schließ-lich
Schwä-bisch
schwä-bi-sche
Schwä-bi-sche
schwä-bi-schen
schwa-che
Schwarz-wald
Schwei-zer
Schwer-punkt
Se-kun-där-um-laut
Sen-se-be-zirk
Se-quen-zier-ungs-be-schrän-kung-en
Se-ria-li-sier-ungs-be-schrän-kung-en 
Sim-pli-fi-zie-rung
so-wie
so-wohl
so-zio-lin-gu-is-ti-schen
Spen-cer
Spe-zi-al-fäl-le
spe-zi-fisch-sten
spe-zi-fi-ziert
Spra-che
Spra-chen
Sprach-ge-mein-schaft
Sprach-ge-mein-schaf-ten
Sprach-ge-schich-te
Sprach-imi-ta-tio-nen
Sprach-in-sel
Sprach-kon-takts
Sprach-wan-del
Sprach-wissenschaftlers
sprach-wis-sen-schaft-li-cher
Sprech-akt-opera-tor
Sprech-akt-opera-tor-en
Sprech-er-be-kennt-nis 
Stamm-for-men
Stamm-mo-di-fi-ka-ti-on
Stan-dar-di-sie-rungs-pro-zess
Standard-Ost-jid-disch
Stan-dard-spra-che
stär-ker
steht
stel-len
stell-ver-tre-tend
Step-pe
steu-ert
struk-tu-rel-le
struk-tu-rel-ler
Stutt-gar-ter
Sub-stan-tiv
Sub-stan-tiv-
sued-oes-tli-chen
Sum-me
Süß-kram
Syn-kre-tis-men
Syn-kre-tis-mus
Syn-kre-tis-mus-paa-res
syn-tak-ti-schen
syn-tak-ti-scher
syn-te-thi-schen
Sys-tem
sy-ste-ma-tisch
sys-te-ma-ti-sche
sys-te-ma-ti-schem
Sys-te-ma-ti-sie-rungs-ar-beit
Szcze-pa-niak
Ta-bel-le
Ta-bel-len
täg-li-chen
Ta-xo-no-mie
ta-xo-no-mi-sche
Teil-a-na-ly-sen
Teil-dis-zi-pli-nen
the-o-re-ti-sche
The-o-rie-bil-dung
trenn-te
Ü-ber-gang
ü-ber-neh-men-den
Ü-ber-set-zung
Ü-ber-spe-zi-fi-ka-ti-on
um-lau-ten
un-ab-hän-gig
un-be-stimm-te
un-be-stimm-ten
un-ga-ri-schen
un-ge-fähr
U-ni-fi-ka-tion
Un-ter-ka-pi-teln
un-ter-schie-den
un-ter-schied-li-che
un-ter-schied-li-chem
un-ter-schied-li-chen
un-ter-schied-li-cher
Un-ter-spe-zi-fi-ka-tion
Un-ter-stüt-zung
un-ter-such-ten
Un-ter-su-chungs-ge-biet
Ur-in-do-ger-ma-ni-sche
Ut-rum
Va-ri-a-ti-on
Ver-all-ge-mei-ner-ung
ver-an-schau-licht
Verb-cluster
Ver-bin-dung-en
Verb-kette
Verb-ketten
Verb-kom-plex
Verb-partikel
Verb-par-ti-kel
Verb-partikeln
Verb-par-ti-keln
Verb-phra-se
Verb-zweit-stel-lung
Ver-deut-li-chung
ver-ein-fach-te
Ver-gleich
Ver-gleich-bar-keit
Ver-glei-che
Ver-gleichs
ver-gleichs-wei-se
Ver-laufs-mus-ter
ver-letzt
ver-or-ten
ver-schie-de-ne
ver-schie-de-nen
ver-schie-den-en
ver-schie-de-ner
ver-stan-den
Ver-tei-lung
ver-tre-te
Ver-tre-ters
ver-wand-te
Ver-wen-dungs-un-ter-schied
Ver-wen-dungs-un-ter-schiede
ver-wer-fen
Ver-zer-ren
Vis-per-ter-mi-nen
VO-Grund-wort-stel-lung
vo-ka-lisch
voll-stän-di-ges
von-ei-nan-der
Vor-be-rei-tung
vor-ge-stellt
vor-han-den
vor-kom-men
vor-kommt
Vor-schlä-ge
wahr-neh-mungs-psy-cho-lo-gi-scher
wei-sen
wei-ten
wei-ter
Wei-te-ren
wel-che
wel-chen
we-ni-ger
we-sent-lich
we-sent-lich-en
wes-halb
west-jid-disch
West-jid-disch-Projekt
West-mittel-deutsch
wich-tig
wie-der-ho-len
Will-kop
wis-sen-schafts-ge-schicht-li-chen
wo-durch
Wohl-ge-formt-heit
Wort-ar-ten
Wör-ter
wur-de
wür-de
Wur-zel
Wur-zel-al-ter-na-ti-o-nen
Wur-zeln
Zel-len
Zu-falls-prin-zip
zu-ge-wie-sen
Zu-ge-zo-ge-nen
zugrunde-lie-gen-de
zu-rück-er-o-bert
zu-sam-men-fal-len
Zu-sam-men-ge-fasst
zu-sam-men-ge-nom-men
Zu-sam-men-hang
zu-sätz-lich
Zu-schrei-bung
Zu-stand
zu-tref-fen
zwei-tens
zwi-schen
}